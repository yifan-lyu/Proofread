\documentclass[12pt, letterpaper]{article}
\usepackage{setspace}
\setstretch{1.3} % control for line spacing without affecting footnote
\usepackage{preamble}
\renewcommand{\abstractname}{}
\begin{document}
\providecommand{\titlename}{Funding Application}
\title{\titlename}
\author{
    {Yifan Lyu\thanks{Stockholm School of Economics, Department of Economics, E-mail: \textcolor{brickred}{yifan.lyu@phdstudent.hhs.se}}
    }
}
\date{\monthyeardate\today}
\maketitle %This command is added to show title, author and date

\vspace{-0.5cm}

%%%%%%%%%%%%%%%%%%%%%%%%%%%%
% Main Document
%%%%%%%%%%%%%%%%%%%%%%%%%%%%

Yifan is a fourth year PhD student specialising in family and labour economics. He uses structural econometric methods combined with real-world micro data to understand how individuals and families make decisions where economic incentives and social norms coexist. His work addresses policy-related questions with broader societal impacts.

%--------------------------- [family planning policy]


In his first chapter, `Family Planning Policy, Human Capital, and Demographics', Yifan delves into the long-term effects of China's fertility policy on demographics and human capital. Using an heterogenous agent overlapping generation model with endogenous fertility and human capital investment, the study investigates the debated topic of whether having fewer children leads to a more educated next generation \citep{Zhang2017}. The model generates patterns that match the level of education and family size observed in the data. The finding shows that overall human capital is contingent on the fertility patterns of different socioeconomic groups, i.e., composition effect. For example, policy-induced rise in overall human capital of children was mainly driven by the composition effect instead of the quantity-quality trade-off. When composition effect dominates, a relaxation in fertility control does not necessarily lower overall human capital. Additionally, Yifan's work predicts future population trends under counterfactual policies, providing insights into the complex relationship between policy, human capital, and population dynamics.

%--------------------------- [Singlehood stigma]
In his study, `Singlehood Stigma', Yifan investigates the social stigma faced by single women and its implications in the marriage market. The relationship between a woman's age at first marriage and the quality of her spouse (in terms of earnings and education) follows a hump-shaped pattern suggesting that there exists age penalty in the marriage market for women. In contrast, this decline in spousal quality for men starts later and is much gentler. This could be related to women’s depreciating reproductive capital \citep{Low2023} yet this channel does not explain well why developed countries observed less penalty while developing countries observed more. To disentangle the declining reproductive capital against social stigma, this work proposes a static marriage market model that rationalises the observed marriage market penalty for women while allowing us to explore the existence of a social stigma attached to single women older. Given that women are less likely to continue investing in their human capital after marriage, social stigma to get married at an earlier age may result in lower human capital for women. This study helps pin down the role of social norms and societal pressures that influence women's marital decisions and the subsequent impact on their educational and career trajectories. 

%--------------------------- [Learning (JMP)]

Apart from social norms, economic incentive is another aspect that influence marriage decision. Yifan's job market paper, `Economic Incentive and Learning in the Marriage Market: Evidence from Sweden’s Pension Reform', utilises Sweden's survivors' insurance reform to explore how changes in the survival pension in the very long run affect the timing of moving from cohabitation to marriage and hence shed light on the complementarity and substitution between marriage and cohabitation. The reform, which altered survivor pension benefits, prompted a change in the timing of marriage among cohabiting couples \citep{Persson2020}. In general, it is common for couples to cohabit before marrying to gather information about the quality of their match and avoid a costly divorce.  This study exploits different duration of cohabitation and their subsequent divorce rate to understand the role of cohabitation. This is the first paper exploring the information gathering and learning process during cohabitation. Using an intra-household bargaining model with endogenous marriage, cohabitation and divorce extended from \cite{Voena2015}, this study sheds light on the role of learning during cohabitation and establish its influence on matching quality and hence marital stability. The result could help policy makers to implement a variety of policies, e.g., how to tax married and cohabiting families, how to avoid costly divorce, and evaluate the de facto marriage law to enhance social welfare.




%--------------------------- [Side project]
Yifan is also passionate in understanding how environment shapes individual performances. Two side projects at the juncture of management science and economics utilise novel date to find answers. The first project, \textit{Loss Aversion: Evidence from expert chess tournaments}, collects data on every single moves in games from chess tournaments to answer how monetary incentives affect players' performances. The result shows the quality of games played by male players is significantly better when they are threatened to lose prize money in the event of a defeat, than when they are motivated by gaining an equivalent sum in the event of a victory. However, this pattern does not hold for female players. This evidence adds up to the literature on how performance by gender under pressure responds differently to economic incentives. The second project, \textit{Local Labor Market Frictions and Platform-Based Entrepreneurship}, examines the relationship between local labor market conditions and the performance of platform-based entrepreneurs. Using data on sales volumes and revenues on millions of U.S.-based shop owners on Etsy, an online marketplace, this research highlights how labor market frictions lead to superior entrepreneurial performance by attracting high-quality individuals to self-employment. The finding contribute to understanding the strategic decisions of entrepreneurs in digital platforms and the interaction between labour market dynamics and entrepreneurial strategies.
%--------------------------- [Summary]






%------------------------------------

%\clearpage %start a new page
\addcontentsline{toc}{section}{References}
\bibliography{Fertility.bib,Marriage.bib}

\end{document}

